\chapter{Uvod}

\qquad U ovom radu je predstavljeno upravljanje električnim vozilom (slika \ref{car:buggy}) sa pogonom na sva četiri točka. Proizvođač \href{https://www.qsmotor.com/product/8000w-car-motor/}{motora} je kompanija \textit{QSMOTOR}. Snaga motora je $8000W$, a standardni napon napajanja iznosi $72V$. Svaki od motora je spojen na odgovarajući pretvarač/\href{https://kellycontroller.com/shop/kls-h/}{invertor}. Motori su trofazni i sadrže dva seta Hallovih senzora za određivanje brzine i pozicije. Datasheet motora je dat u prilogu \ref{motor_datasheet}, a dokumentacija pretvarača \textit{KLS7275H} u \cite{kls7275h}. Razvoj upravljačkog prototipa će biti realiziran pomoću \href{https://www.dspace.com/en/inc/home/products/hw/microlabbox.cfm?fbclid=IwAR08_hHwsXPVRs6ng2DLSU5HA3vDNzpBa9CMpO8DWlSQ1DXPK58BkLmoiRE}{dSpace} sistema.

\begin{figure}
\begin{center}
\includegraphics[scale=0.5]{slike/electric_car.jpg}
\end{center}
\caption{Električno $4x4$ vozilo}
\label{car:buggy}
\end{figure}

\section{Motor}

\qquad Motor je trofazni sa $16$ pari polova, nominalne snage $8000W$. Za mjerenje brzine i pozicije motora, dostupna su dva seta Hallovih senzora. Moguće je mjeriti i temperaturu motora pomoću temperaturnog senzora $KTY83/122$. Motor sadrži dva konektora, čiji pinovi odgovaraju pinovima na \textit{DJ7061Y-2.3-21} konektoru energetskog pretvarača. Svaki od konektora daje informacije sa Hallovog i temperaturnog senzora.

\section{Papučica gasa}

\qquad Brzinu motora je moguće zadavati pomoću papučice gasa. Model koji je odabran ima serijski broj \textit{JKH-005-A-65}. Prema specifikaciji proizvođača \textit{SAYOO}, datoj u \cite{jkh005a65}, dužina kabla spojenog na kočnicu iznosi $65cm$. U kablu se nalazi 5 žica raspoređenih na dva konektora - jedan sa tri, drugi sa četiri žice. Raspon ulaznog i izlaznog napona papučice gasa iznosi od $0$ do $5V$. Ovaj element se konceptualno ponaša kao otpornik sa klizačem koji je spojen na napon napajanja. Izlazni napon otpornika ovisi o položaju klizača, odnosno u ovom slučaju - položaja papučice gasa. Pinovi, prikazani na slici (\ref{fig:gas}), imaju funkcije date u nastavku.

\begin{itemize}
	\item (1) - plus napon napajanja ($5V$)
	\item (2) - minus napon napajanja ($0V$, masa)
	\item (3) - izlazni napon
	\item (4) - izlazni kontakt prekidača
	\item (5) - ulazni kontakt prekidača
\end{itemize}

\begin{figure}
\centering
\includegraphics[scale=1]{slike/foot_throttle.png}
\caption{Pinovi papučice gasa}
\label{fig:gas}
\end{figure}

\section{Pretvarač}

\qquad Energetski pretvarač \textit{KLS7275H} proizvođača \textit{Kelly} sadrži 5 digitalnih ulaza: prekidače za gas i kočnicu, prekidače za kretanje naprijed i nazad, te prekidač za \textit{boost} način rada. Dostupna su 3 analogna ulaza: gas, kočnica i temperatura motora. Opseg ulaznog napona može varirati od $0$ do $5V$. Za upravljanje je moguće koristiti i palicu (\textit{engl. joystick}), pri čemu pozicija palice određuje zadanu brzinu i smjer kretanja. \textit{Cruise} režim rada, koji je također dostupan, podrazumijeva zadržavanje zadane brzine vrtnje motora sve dok se ne zada nova brzina ili aktivira kočnica. Pretvarač podržava povezivanje putem \textit{CAN} mreže.

\subsection{Pinovi energetskog pretvarača}

\qquad \textit{KLS7275H} kontroler posjeduje tri konektora prikazana na slici (\ref{fig:kellypins}) i 22 pina. Pored standardnih pinova za napajanje, pretvarač posjeduje pinove za prikupljanje informacija sa motora (Hallov senzor i temperaturni senzor), kao i pinove za definiranje smjera i brzine vrtnje motora. Žica spojena na odgovarajući pin je označena jedinstvenom bojom i brojem. Funkcije pinova su date u nastavku.

\begin{itemize}
	\item Konektor \textit{DJ7091Y-2.3-11}
	\begin{itemize}
		\item (14) REV-SW - prekidač za kretanje nazad
		\item \hphantom{0}(6) GND - minus napona napajanja, povratni signal, masa
		\item (12) FWD - prekidač za kretanje naprijed
		\item (11) 12V - naponski izvor od $12V$
		\item (25) 12V Brake - ručna kočnica
		\item (22) ECO - prekidač za štedljivi način rada
		\item (33) CAN-H - \textit{high} pin za CAN komunikaciju
		\item \hphantom{0}(7) PWR - plus napona napajanja pretvarača
		\item (34) CAN-L - \textit{low} pin za CAN komunikaciju
	\end{itemize}
	\item Konektor \textit{DJ7091Y-2.3-21}
		\begin{itemize}
		\item (15) FOOT-SW - prekidač za gas
		\item \hphantom{0}(3) Throttle - analogni ulaz za gas ($0-5V$)
		\item (20) GND - minus napon napajanja, povratni signal, masa
		\item \hphantom{0}(8) Meter - kopija signala sa Hallovog senzora
		\item \hphantom{0}(4) 5V - naponski izvor od $5V$
		\item \hphantom{0}(2) Brake-AN - \textit{boost} funkcija ili analogni ulaz za regenerativni tip kočenja
		\item (11) 12V - naponski izvor od $12V$
	\end{itemize}
	\item Konektor \textit{DJ7061Y-2.3-21}
		\begin{itemize}
		\item (21) GND - minus napon napajanja, povratni signal, masa
		\item \hphantom{0}(1) Temp - temperatura motora
		\item \hphantom{0}(5) 5V - naponski izvor od $5V$
		\item (18) Hall A - signal Hallovog senzora za fazu A
		\item (17) Hall B - signal Hallovog senzora za fazu B 
		\item (16) Hall C - signal Hallovog senzora za fazu C 
	\end{itemize}
\end{itemize}

\begin{figure}
\begin{center}
\includegraphics[scale=1]{slike/kellypins.jpg}
\end{center}
\caption{Pinovi \textit{Kelly KLS7275H} pretvarača}
\label{fig:kellypins}
\end{figure}

\section{Upravljanje motorom}

\qquad Energetski pretvarač/kontroler koristi vektorsku modulaciju (\textit{engl. space-vector modulation}) za upravljanje trofaznim motorom. Za upravljanje cjelokupnim sistemom vozila, potrebno je uključiti dodatni sloj upravljanja, koji će upravljati paralelno sa svim pretvaračima koji su povezani sa motorima.

Za sada će konceptualno biti razmotreno upravljanje jednim motorom. Shema upravljanja je data na sljedećoj strani. Osnovna ideja upravljanja se zasniva na tome da se na pin (3) \textit{KLS7275H} pretvarača dovodi signal u rasponu od $0$ do $5V$, koji će proporcionalno svojoj vrijednosti - određivati željenu brzinu vrtnje motora. Izvor signala gasa može biti potenciometar ili nožna papučica, što se također mora uzeti u obzir prilikom konfiguracije samog pretvarača. U prikazanoj konfiguraciji je potrebno obezbjediti signal koji će na pinu (15) davati informaciju o stanju papučice gasa. Funkcija analognog regenerativnog kočenja se može realizirati na sličan način, pri čemu za razliku od gasa, nije potrebna infromacija da li je papučica aktivirana. Na pinu (25) je moguće realizirati $12V$ ručnu kočnicu.

Korisnik može mijenjati smjer vrtnje motora, odnosno smjer kretanja vozila (naprijed ili nazad) dovođenjem odgovarajuće kombinacije signala na pinove (12) i (14). Na osnovu ulaznih signala, energetski pretvarač pokreće trofazni DC motor bez četkica. Pinovi (16), (17) i (18) služe za prikupljanje signala sa Hallovog senzora na samom motoru. Napajanje Hallovog senzora iznosi $5V$.

\includepdf[pages=-]{dokumenti/proteus upravljanje}


\begin{comment}

\begin{table}[]
\centering
\begin{tabular}{|c|c|c|c|c|c|}
\hline
\multicolumn{2}{|c|}{Configuration} & \multicolumn{3}{c|}{Pin Status} & \multirow{2}{*}{Running Status} \\ \cline{1-5}
Forward Switch & Foot Switch & FWD-SW (12) & REV-SW (14) & FOOT (15) &  \\ \hline
\multirow{4}{*}{Enable} & \multirow{4}{*}{Disable} & OFF & OFF & X & Neutral \\ \cline{3-6} 
 &  & OFF & ON & X & Reverse \\ \cline{3-6} 
 &  & ON & OFF & X & Forward \\ \cline{3-6} 
 &  & ON & ON & X & Neutral \\ \hline
\multirow{4}{*}{Disable} & \multirow{4}{*}{Enable} & X & OFF & OFF & Can't operate \\ \cline{3-6} 
 &  & X & ON & OFF & Can't operate \\ \cline{3-6} 
 &  & X & ON & ON & Reverse \\ \cline{3-6} 
 &  & X & OFF & ON & Forward \\ \hline
\multirow{2}{*}{Disable} & \multirow{2}{*}{Disable} & X & OFF & X & Forward \\ \cline{3-6} 
 &  & X & ON & X & Reverse \\ \hline
\end{tabular}
\caption{Konfiguracija pretvarača}
\label{tab:konfiguracija}
\end{table}

\end{comment}

\section{Konfiguracija parametara pretvarača}

\qquad Parametre \textit{KLS7275H} kontrolera je moguće podešavati pomoću PC ili Android uređaja. Veza sa PC uređajem se ostvaruje preko \textit{USB-RS232} konektora. Softver za konfiguraciju je dostupan na zvaničnoj \href{https://kellycontroller.com/shop/kls-h/}{stranici} proizvođača. Dijaloški okvir programa je prikazan na slici (\ref{fig:configuration}). Podešavanja vezana za motor i samo vozilo koje će motor pogoniti kao dio sistema su odvojena. Informacije kao što su ime modula, serijski broj, verzija softvera i napon pretvarača su dostupna samo za čitanje.

\begin{figure}
\centering
\includegraphics[width=\textwidth]{slike/configuration.jpg}
\caption{Program za konfiguraciju energetskog pretvarača}
\label{fig:configuration}
\end{figure}

\section{Operacija identifikacije ugla}

\qquad Prije puštanja određenog motora u rad, potrebno izvršiti proceduru \textit{identifikacije ugla}. Ova operacija se izvršava korištenjem aplikacije na računaru koji je povezan sa energetskim pretvaračem preko \textit{USB-RS232} konektora. Shema spajanja je data na slici (\ref{fig:angleident}). Pretvarač je spojen na motor sa tri faze, a sa motora se prenosi signal sa Hallovog senzora. Vanjski element za zadavanje gasa također treba povezati sa pretvaračem, pri čemu taj element može jednostavno biti potenciometar ili papučica gasa.

\begin{figure}
\centering
\includegraphics[width=\textwidth]{slike/angleident.jpg}
\caption{Shema spajanja za operaciju identifikacije ugla}
\label{fig:angleident}
\end{figure}

Komponenta koja se može koristiti u tu svrhu je \textit{Controller Control Box}, prikazana na slici (\ref{fig:controlbox}). Shema spajanja sa kontrolerom preko 14-pinskog konektora je data u prilogu \ref{control}, pri čemu je za realizaciju identifikacije ugla dovoljno koristiti opciju u kojoj se \textit{PWR} prekidač na kontrolnoj kutiji koristi za \textit{Key switch} na slici (\ref{fig:angleident}) i konfiguracija \textit{Throttle} potenciometra kao element zadavanja gasa.

\begin{figure}
\centering
\includegraphics[scale=0.1]{slike/controlbox.jpg}
\caption{Controller Control Box}
\label{fig:controlbox}
\end{figure}

Pri povezivanju svih elemenata potrebno je da \textit{PWR} prekidač, odnosno\textit{Key switch} bude u otvorenom stanju. Nakon što se preuzme softver za konfiguraciju i instalira na PC, prekidač za napajanje je potrebno zatvoriti i pokrenuti instalirani softver. Kada se otvori dijaloški prozor programa (slika \ref{fig:configuration}), potrebno je kliknuti na opciju \textit{Read}, kako bi se učitali trenutni podaci kontrolera.

Parametri koje je potrebno konfigurirati prije procesa identifikacije ugla se nalaze unutar grupe \textit{Motor}, a odnose se na broj polova motora, tip senzora brzine i iznos struje koja se koristi za identifikaciju. Unutar grupe \textit{Vehicle} potrebno je podesiti parametre: \textit{TPS Type} (način zadavanja gasa), \textit{Max Output Frequency} (maksimalna vrijednost frekvencije za faze motora), \textit{Max Speed} (maksimalna brzina motora, izražena u obrtajima po minuti) i \textit{PWM frequency} (frekvencija širinsko-impulsne modulacije, odnosno metode kojom pretvarač upravlja motorima). Treba napomenuti da brzina rotacije motora, prema specifikaciji proizvođača datoj u prilogu \ref{motor_datasheet}, brzina rotacije motora iznosi od 550 do 1200 RPM.

Naredni korak u postupku identifikacije ugla je upisivanje broja \textbf{170} u parametar \textit{Identification Angle} i odabir opcije \textit{Write}. Zatim je potrebno izaći iz programa za konfiguraciju. Isključivanjem dovoda napajanja na nekoliko sekundi i ponovnim spuštanjem \textit{PWR} prekidača, motor započinje kretanje nasumično u oba smjera. Proces identifikacije traje od 2 do 3 minute.

Kada se proces završi, pretvarač će dati $3-2$ kod greške - zujalica (\textit{engl. buzzer}) će najprije dati 3, a zatim 2 brza zvučna signala. Zatim je potrebno opet na nekoliko sekundi isključiti napajanje kontrolera. Nakon ponovnog pokretanja korisničkog sučelja za konfiguraciju kontrolera, potrebno je provjeriti \textit{Identification Angle} parametar, ukoliko je proces identifikacije uspješno izvršen - parametar ima vrijednost 85.

Iako se u procesu identifikacije ugla nije koristio potenciometar za davanje gasa, ovaj element služi za provjeru da li se motor kreće u željenom smjeru. Ukoliko to nije slučaj, potrebno je označiti (\textit{engl. check}) opciju \textit{Change Direction} u programu za konfiguraciju. Da bi se promjena sačuvala potrebno je kliknuti na dugme \textit{Write} i resetirati napajanje.

\section{Model vozila}

\qquad Općeniti model vozila uzima u obzir translacijsko kretanje u $xyz$ koordinatnom sistemu i rotacijsko kretanje oko osa tog koordinatnog sistema (\textit{roll}, \textit{pitch} i \textit{yaw}), kao i rotacijsko kretanje kotača. U slučaju vozila sa četiri točka, tada se radi o modelu sa 13 stepeni slobode, uz zanemarivanje dinamike sistema suspenzije. Koordinatni sistem vozila, prema \href{https://www.iso.org/standard/51180.html}{ISO 8855:2011} standardu je prikazan na slici (\ref{model:coordinate-system}).

\begin{figure}
\centering
\includegraphics[scale=0.5]{slike/model/coordinate-system.png}
\caption{Koordinatni sistem vozila \citep{kissai2019adaptive}}
\label{model:coordinate-system}
\end{figure}

Modeli se mogu razlikovati i po sistemu zakretanja vozila. Različiti sistemi tog tipa su prikazani na slici (\ref{model:steering-systems}). Najćešće korištene konfiguracije podrazumijevaju skretanje bez rotacije kotača, skretanje uz rotaciju prednjih kotača i skretanje uz rotaciju svih kotača. Rotacija se obavlja oko $z$ ose.

\begin{figure}
\centering
\includegraphics[width=\textwidth]{slike/model/steering-systems.jpg}
\caption{Različiti sistemi zakretanja vozila \citep{han2010development}}
\label{model:steering-systems}
\end{figure}

Vozilo, prikazano na slici (\ref{car:buggy}), nema mogućnost rotiranja kotača - te se skretanje realizira zadavanjem različitih vrijednosti momenata/brzina motorima na lijevoj i desnoj strani vozila. Neka se razmatra skretanje udesno, kao na slici (\ref{model:skid-steer}). 

Unutrašnji (\textit{engl. inner}) kotači se nalaze bliže, a vanjski (\textit{engl. outer}) dalje od tačke oko koje se vrši rotacija. U ovom slučaju lijevi točkovi su vanjski, sa brzinom $V_o$, dok su desni točkovi unutrašnji, sa brzinom $V_i$. Vozilo će skrenuti udesno, ukoliko vrijedi da je $V_o>V_i$. Analogno se može izvesti zaključak za slučaj skretanja ulijevo.

\begin{figure}
\centering
\includegraphics[scale=1]{slike/model/skid-steer.jpg}
\caption{Skid steering \citep{shuang2007skid}}
\label{model:skid-steer}
\end{figure}

Cilj ovakvog upravljanja je promjena orijentacije vozila koje se svjesno podstiče na proklizavanje, pa se ovakav način skretanja vozila u literaturi naziva \textit{skid steering}.

\subsection{Jednačine kretanja}

\qquad Shematski prikaz vozila je prikazan na slici (\ref{model:model}). Ovaj model podrazumijeva kretanje po ravnom terenu, pri čemu se tada zanemaruje translacija po $z$-osi i rotacija oko osa $x$ i $y$.

Model sa 7 stepeni slobode, preuzet iz \citep{aslam2014fuzzy}, sastoji se od jednačina (\ref{eq:model-X}) i (\ref{eq:model-Y}) za kretanje u $xy$-ravni, jednačine (\ref{eq:model-rot}) za rotaciju oko $z$-ose i jednačine (\ref{eq:model-wheel}) za rotaciju kotača.

\begin{equation}\label{eq:model-X}
M a_x = M ( \dot{v}_x - \gamma v_y ) = \sum_i F_{x,i} = F_{x,fl}+F_{x,fr}+F_{x,rl}+F_{x,rr}
\end{equation}

\begin{equation}\label{eq:model-Y}
M a_y = M ( \dot{v}_y - \gamma v_x ) = \sum_i F_{y,i} = F_{y,fl}+F_{y,fr}+F_{y,rl}+F_{y,rr}
\end{equation}

\begin{equation}\label{eq:model-rot}
J \dot{\gamma} = l_s (F_{x,fr}+F_{x,rr}-F_{x,fl}-F_{x,rl}) + l_f (F_{y,fl}+F_{y,fr}) - l_r (F_{y,rl}+F_{y,rr})
\end{equation}

\begin{equation}\label{eq:model-wheel}
J_w \dot{\omega}_i = \tau_i - R_w F_{x,i}
\end{equation}

\begin{equation*}
i = fl, fr, rl, rr
\end{equation*}

\begin{figure}
\centering
\includegraphics[width=\textwidth]{slike/model/model.jpg}
\caption{Shematski prikaz vozila \citep{tian2018differential}}
\label{model:model}
\end{figure}

Parametri ovog modela su:

\begin{itemize}
	\item $i$ - oznaka kotača (\textit{fl},\textit{fr},\textit{rl},\textit{rr})
	\item $fl$ - oznaka za prednji lijevi (\textit{engl. front-left}) točak
	\item $fr$ - oznaka za prednji desni (\textit{engl. front-right}) točak
	\item $rl$ - oznaka za zadnji lijevi (\textit{engl. rear-left}) točak
	\item $rr$ - oznaka za zadnji desni (\textit{engl. rear-right}) točak
	\item $M$ - masa vozila
	\item $a_x$ - longitudinalno ubrzanje
	\item $a_y$ - lateralno ubrzanje
	\item $v_x$ - longitudinalna brzina
	\item $v_y$ - lateralna brzina
	\item $\gamma$ - ugaona brzina centra mase vozila
	\item $\omega_i$ - ugaona brzina $i$-tog kotača
	\item $F_x,i$ - longitudinalna sila koja djeluje na $i$-ti točak
	\item $F_y,i$ - lateralna sila koja djeluje na $i$-ti točak
	\item $J$ - moment inercije vozila
	\item $J_w$ - moment inercije točka
	\item $ls$ - udaljenost između točkova na lijevoj/desnoj strani i tačke centra mase
	\item $lf$ - udaljenost između tačke centra mase i prednjih kotača
	\item $lr$ - udaljenost između tačke centra mase i zadnjih kotača
	\item $\tau_i$ - moment $i$-tog kotača
	\item $R_w$ - poluprečnik točka

\end{itemize}
	
\subsection{Kretanje po jednoj osi}

\qquad Ukoliko se razmatra kretanje samo po jednoj osi, moguće je izvesti nešto jednostavniji model kretanja. Neka se analizira kretanje naprijed-nazad po $x$-osi. Slijedi da nema rotacije oko $z$ ose ($\gamma=0$), kao ni translacije na pravcu $y$-ose ($v_y=0$). Sada jednačine (\ref{eq:model-Y}) i (\ref{eq:model-rot}) nisu više predmet razmatranja, jer nemaju utjecaj na pretpostavljeni tip kretanja.

Za kretanje naprijed-nazad se može uvesti pretpostavka da su sve longitudinalne sile jednake, kao što je navedeno u relaciji (\ref{eq:model-jednake-sile}). S obzirom na to da se u slučaju kretanja po pravcu svim motorima zadaje ista vrijednost momenta, možemo pretpostaviti da su momenti svih motora jednaki, kao što je navedeno u relaciji (\ref{eq:model-jednaki-momenti}). Isti se zaključak može izvesti i za ugaone brzine točkova, što je dato u relaciji (\ref{eq:model-jednake-brzine}).

\begin{equation}\label{eq:model-jednake-sile}
F_{x,i}=F_{x,j}=F_X
\end{equation}

\begin{equation}\label{eq:model-jednaki-momenti}
\tau_i=\tau_j=\tau_w
\end{equation}

\begin{equation}\label{eq:model-jednake-brzine}
\omega_i=\omega_j=\omega_w
\end{equation}

\begin{equation*}
i,j = fl, fr, rl, rr
\end{equation*}

Uvrštavanjem relacija (\ref{eq:model-jednake-sile}), (\ref{eq:model-jednaki-momenti}) i (\ref{eq:model-jednake-brzine}) u jednačine (\ref{eq:model-X}) i (\ref{eq:model-wheel}), dobiva se model kretanja vozila po pravcu na ravnom terenu, dat jednačinama (\ref{eq:model-easy-vx}) i (\ref{eq:model-easy-om}).

\begin{equation}\label{eq:model-easy-vx}
M \dot{v}_x = 4 \cdot F_X
\end{equation}

\begin{equation}\label{eq:model-easy-om}
J_w \dot{\omega_w} = \tau_w - R_w F_X
\end{equation}

\subsection{Model gume}

\qquad Za određivanje lateralnih i longitudinalnih sila, potrebno je poznavati model gume. Postoji nekoliko različitih matematičkih modela koji rješavaju ovaj problem, kao što su npr. \textit{UniTire} ili \textit{Fiala} modeli \citep{ma2017design}.

\subsubsection{Čarobni model gume}

\qquad Najzastupljeniji model u literaturi je poznat pod nazivom \textit{čarobni model gume} (\textit{engl. magic tire model}) \citep{pacejka2005tire} ili \textit{čarobna formula}. Općeniti oblik čarobne formule je dat jednačinama (\ref{eq:guma-Y}), (\ref{eq:guma-x}) i (\ref{eq:guma-y}).

\begin{equation}\label{eq:guma-Y}
Y(X) = y(x) + S_V
\end{equation}

\begin{equation}\label{eq:guma-x}
x = X + S_H
\end{equation}

\begin{equation}\label{eq:guma-y}
y = D \cdot \sin \{ C \cdot \arctan \{B \cdot x-E \cdot [B \cdot x- \arctan (B \cdot x)] \} \}
\end{equation}

Funkcija $Y(X)$ se može koristiti za modeliranje lateralne sile, longitudinalne sile ili ugaonog momenta. Varijabla $X$ može imati karakter omjera uzdužnog proklizavanja (\textit{engl. longitudinal slip ratio}) $\lambda$ ili ugla bočnog klizanja (\textit{engl. side-slip angle}) $\alpha$.

$B$, $C$, $D$ i $E$ su parametri koji se određuju eksperimentalno, a parametri $S_H$ i $S_V$ služe za definiranje \textit{offseta} sila. Interpretacija parametara je data na slici (\ref{model:magic-formula}).

\begin{figure}
\centering
\includegraphics[width=\textwidth]{slike/model/magic-formula.png}
\caption{Interpretacija parametara \textit{čarobne formule} \citep{lidfors2020direct}}
\label{model:magic-formula}
\end{figure}

\subsection{Određivanje sile $F_X$}

\qquad Neka su offseti sila $S_H=S_V=0$. Tada se longitudinalna sila $F_x(\lambda)$ može odrediti pomoću jednačine (\ref{eq:guma-fx}). Omjer uzdužnog proklizavanja $\lambda$ je definiran sa (\ref{eq:guma-lambda}) \citep{asiabar2019direct}.

\begin{equation}\label{eq:guma-fx}
F_x(\lambda) = D \cdot \sin \{ C \cdot \arctan \{B \cdot \lambda - E \cdot [B \cdot \lambda - \arctan (B \cdot \lambda )] \} \}
\end{equation}

\begin{equation}\label{eq:guma-lambda}
\lambda = 
\left\{
\begin{array}{ccc}
-1 + \frac{R_w \cdot \omega_w}{v_x} & , & R_w \omega_w \leq v_x \\
1 - \frac{v_x}{R_w \cdot \omega_w} & , & R_w \omega_w \geq v_x
\end{array}
\right.
\end{equation}

Treba napomenuti da u opštem slučaju $\lambda$ ovisi o brzini centra mase točka i geometriji samog vozila, ali da se u slučaju linearnog kretanja brzina centra kotača može aproksimirati longitudinalnom brzinom vozila $v_x$. Vrijednosti koeficijenata $B$, $C$, $D$ i $E$ su preuzeti iz \citep{li2015novel} i imaju vrijednosti date u (\ref{eq:guma-bcde}).

\begin{equation}\label{eq:guma-bcde}
B = 11,45 \quad C = 1,62 \quad D = 4243 \quad E = 0,48
\end{equation} 
